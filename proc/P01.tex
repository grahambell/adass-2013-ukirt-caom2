% This is the aspauthor.tex LaTeX file
% Copyright 2010, Astronomical Society of the Pacific Conference Series

\documentclass[11pt,twoside]{article}
\usepackage{asp2010}

\resetcounters

\bibliographystyle{asp2010}

\markboth{Bell et al.}{A New Archive of UKIRT Legacy Data at CADC}

\begin{document}

\title{A New Archive of UKIRT Legacy Data at CADC}
\author{Graham~S.~Bell,$^1$ Malcolm~J.~Currie,$^1$ Russell~O.~Redman,$^2$ Maren~Purves,$^1$ and Tim~Jenness$^{1,3}$
\affil{$^1$Joint~Astronomy~Centre, 660~N.~A`oh\={o}k\={u}~Place, Hilo, HI, 96720, U.S.A.}
\affil{$^2$Canadian Astronomy Data Centre, 5071 West Saanich Road, Victoria, BC, V9E 2M7, Canada}
\affil{$^3$Department of Astronomy, Cornell University, Ithaca, NY, 14853, U.S.A.}}

\begin{abstract}
A new archive of raw data from the United Kingdom Infrared
Telescope (UKIRT) is being constructed at the Canadian Astronomy
Data Centre (CADC).  It contains all data available apart from
that from the Wide Field Camera (WFCAM), which is already
accessible from the WFCAM Science Archive.

The desire was to archive the data in as close to its original raw
format as possible.  Where the data followed our current convention
of having a single data file per observation, it was archived
without alteration, except for minor fixes to headers of data in
FITS format to allow it to pass fitsverify and be accepted by CADC.
Some of the older data comprised multiple integrations in separate
files per observation, stored in either NDF or DST format. These
were placed inside HDS container files, and DST files were rearranged
into NDF format.

The process of ingesting the metadata into the CAOM-2 repository
was broken into two stages.  First the headers were read from
the data files and stored in an intermediate MongoDB database.
Then the ingestion process uses this database to generate an XML
file for each observation and sends this file to the repository.

The final stage is to run the ORAC-DR pipeline to generate reduced
data products.  The intermediate database can be used to identify
groups of observations to be processed together and to supply
additional data where the original headers were insufficient.
\end{abstract}

\section{Introduction}

\ldots

CAOM-2 \citep{2012ASPC..461..339D}

WSA \citep{2008MNRAS.384..637H}

\ldots

A new archive of raw data from the United Kingdom Infrared
Telescope (UKIRT) is being constructed at the Canadian Astronomy
Data Centre (CADC).

It excludes data from the Wide Field Camera (WFCAM),
since it is already available in the WFCAM Science Archive in
Edinburgh.

Data archived in as close to its original raw format as possible.

Includes raw and reduced data.

\articlefigure[]{P01_f1}{p01:fig:flowchart}{A flowchart.}

\subsection{Instruments}

The archive contains data from UKIRT's cassegrain instruments
listed in table~\ref{p01:tab:instruments}.

\begin{table}[!ht]
\caption{Instruments with no stop}
\smallskip
\begin{center}
\begin{tabular}{lcr}
\tableline
\noalign{\smallskip}
Instrument & Type & Wavelength \\
\noalign{\smallskip}
\tableline
\noalign{\smallskip}
CGS3 & spectrometer  & 10--20\,\micron \\
CGS4 & spectrometer & 1--5\,\micron \\
IRCAM3 & imager & 1--5\,\micron \\
Michelle & imager / spectrometer & 8--25\,\micron  \\
UFTI & imager & 1--2.5\,\micron \\
UIST & imager / spectrometer with integral field unit & 1--5\,\micron  \\
\noalign{\smallskip}
\tableline
\end{tabular}
\end{center}
\label{p01:tab:instruments}
\end{table}

\subsection{CAOM-2}

CAOM-2 is the next generation of the Common Archive Observation Model
currently being developed at CADC.

Standardizes the core parts of the archive
between various ``collections''.

Uses a single ``advanced search'' interface for all collections.

UKIRT added as a new collection.

\section{Transfer of Data to CADC}

The raw data were transferred to the archive at CADC
using the
Electronic Transfer system \citep{2005ASPC..347..647M}.
The files were copied to a ``new'' directory in a
staging area where the system agent would
detect them and automate their transfer to the
archive, deleting them when this was complete.
Any files rejected by the agent were instead moved
to one of several ``reject'' directories depending
on the reason.

As the files were staged for transfer,
filenames were standardized following our current convention.
In addition,
some minor adjustments were made,
depending on the format of the file.

\altsubsubsection*{FITS}

Some early raw data from UFTI was written in FITS format.
For transfer to the archive, these files were
required to pass
\texttt{fitsverify},
and this required a few fixes to the headers.
One example was the removal of \texttt{Z} characters
previously used to indicate UTC in date headers.

\altsubsubsection*{NDF}

Much of the data were stored directly in
Starlink extensible N-dimensional Data Format
\citep[NDF, see for example,][]{P91_adassxxiii},
which itself is based on the
Starlink Hierarchical Data System (HDS).
In cases where an observation was recorded as
multiple integrations in separate files,
these were stored in a single HDS container file.

\altsubsubsection*{DST}

Data stored in the older Figaro Format
--- which is also based on HDS ---
were converted to NDF.
As for raw NDF files, multiple integrations
were stored in HDS container files.

\section{Ingestion of Metadata into CAOM-2}

\ldots

Metadata ingestion process split into two parts:

\begin{itemize}
\item Read headers from data files into an intermediate database.

\item Write XML files using the \texttt{pyCAOM2} library, and send to the repository.
\end{itemize}

Using \texttt{MongoDB} for the intermediate database allowed the
headers to be stored without prior knowledge of their structure.

Size of header database: 3.8\,GiB

\section{Data Reduction Pipeline}

ORAC-DR \citep{1999ASPC..172...11E,2008AN....329..295C}

HdrTrans \citep[section 2.2]{2008AN....329..295C}

JSA \citep{2011ASPC..442..203E}

\ldots

Plan to generate reduced data products using the ORAC-DR pipeline.
ORAC-DR already been used for these instruments (except CGS3).

Will use similar infrastructure at CADC as is already
used in the JCMT Science Archive.

Reduce a complete night at once allowing
ORAC-DR to find the required calibrations.

Currently testing the ORAC-DR on a range of data.
For observations pre-dating its introduction:

Some header changes can be handled by updating the
\texttt{Astro::FITS::HdrTrans} module.

Other information such as grouping and recipe names
must be provided, based on the intermediate database.

\articlefigure[angle=270,scale=0.5]{P01_f2}{p01:fig:hh111}{Example image.}

\section{Conclusions}

\ldots

\bibliography{P01}

\end{document}
