\documentclass[a0paper,portrait]{baposter}
\usepackage{relsize}
\renewcommand\RSsmallest{2pt}
\usepackage{mathpazo}
\usepackage{graphicx}
\usepackage{verbatim}
\usepackage{enumitem}
\usepackage{setspace}
\usepackage{siunitx}
\usepackage{color}
\usetikzlibrary{backgrounds}

\definecolor{u2cgreen}{RGB}{0,60,0}
\definecolor{u2cyellow}{RGB}{255,255,0}
\definecolor{u2cltyellow}{RGB}{255,255,255}
\definecolor{listingfaint}{gray}{0.70}

\setitemize[0]{leftmargin=10pt,labelindent=0pt}
%\setenumerate[0]{leftmargin=15pt,itemindent=0pt}

\begin{document}
\begin{poster}{
    columns=3,
    background=none,
    eyecatcher=false,
    textborder=roundedsmall,
    headerborder=closed,
    headershape=smallrounded,
    headershade=shadeTB,
    boxshade=none,
    borderColor=u2cgreen,
    headerColorOne=u2cltyellow,
    headerColorTwo=u2cyellow,
    linewidth=1
}{

}{\raggedright \smaller
A New Archive of UKIRT Legacy Data at CADC
}{
\vspace{0.1cm} \\
Graham Bell,$^1$ Malcolm Currie,$^1$ Russell Redman,$^2$ Maren Purves,$^1$ Tim Jenness$^{1,3}$
\vspace{0.1cm} \\
\smaller[2]
\textsl{$^1$ Joint Astronomy Centre, Hilo HI} \\
\textsl{$^2$ Canadian Astronomy Data Centre, Victoria BC} \\
\textsl{$^3$ Cornell University, Ithaca NY}
}{
\begin{tabular}{r}
\includegraphics[width=2cm]{jaclogo} \\
\includegraphics[width=4cm]{stfclogo}
\end{tabular}
}

\headerbox{Introduction}{name=intro,column=0,row=0}{
    \raggedright
    A new archive of raw data from the United Kingdom Infrared
    Telescope (UKIRT) is being constructed at the Canadian Astronomy
    Data Centre (CADC).

    \begin{itemize}
        \item Excludes data from
        the Wide Field Camera (WFCAM),
        since it is already available
        in the WFCAM Science Archive in
        Edinburgh.

        \item Data archived in as close to its original raw format as possible.

    \end{itemize}

    Illustration of the processes involved in preparing the archive:

    \quad

    \includegraphics[scale=0.5]{../gfx/flow-crop}

    \quad

    \vspace{2.0in}

    \begin{centering}
    \begin{tabular}{ll}
    Total number of observations: & 1,182,507 \\
    Raw data size: & 2.147\,TiB
    \end{tabular}
    \end{centering}
}

\headerbox{CAOM-2}{name=caom2,column=1,row=0}{
    \raggedright

     CAOM-2 is the next generation of the
     Common Archive Observation Model
     currently being developed at CADC.

     \begin{itemize}
         \item Standardizes the core parts of the archive
               between various ``collections''.

         \item UKIRT added as a new collection.

         \item Includes raw and reduced data.

         \item Uses a single ``advanced search'' interface for all collections.
     \end{itemize}
}

\headerbox{E-Transfer to CADC}{name=etransfer,column=1,span=2,below=caom2}{
    \raggedright

    \begin{itemize}
    \item Filenames standardized following our current convention.

    \item Minor changes made to files as they were staged for transfer,
    depending on format:
    \end{itemize}

    \begin{centering}
    \begin{tabular}[t]{ccc}
    \textbf{FITS} & \textbf{NDF} & \textbf{DST} \\

    \parbox[t]{1.8in}{\begin{itemize}[label={--},topsep=0pt]
    \raggedright
    \item
    Header fixes to pass \texttt{fitsverify},
    e.g. removal of trailing \texttt{Z} in date headers.
    \end{itemize}}

    &

    \parbox[t]{1.8in}{\begin{itemize}[label={--},topsep=0pt]
    \raggedright
    \item
    If multiple integrations are separate files,
    store in an HDS container file.
    %with a standardized filename.
    \end{itemize}}

    &

    \parbox[t]{1.8in}{\begin{itemize}[label={--},topsep=0pt]
    \raggedright
    \item
    Rearrange to NDF format.

    \raggedright
    \item
    Store multiple integrations in HDS container files.
    \end{itemize}}

    \end{tabular}
    \end{centering}

    \quad

    \quad

    \smaller[2]
    \textbf{HDS}: Starlink Hierarchical Data System. \\
    \textbf{NDF}: Starlink Extensible N-Dimensional Data Format (an HDS format, see also poster P91). \\
    \textbf{DST}: Figaro Format (an HDS format). \\
}

\headerbox{UKIRT Instruments}{name=inst,column=2,row=0,above=etransfer}{
    \raggedright
    The archive contains data from UKIRT's cassegrain instruments:

    {\center
    \begin{tabular}{cc}
    \bf CGS3                                           & \bf Michelle                                              \\
    \smaller[2] 10--\SI{20}{\micro\meter} spectrometer & \smaller[2] 8--\SI{25}{\micro\meter} imager/spectrometer  \\
                                                       &                                                           \\
    \bf CGS4                                           & \bf UFTI                                                  \\
    \smaller[2] 1--\SI{5}{\micro\meter} spectrometer   &  \smaller[2] 1--\SI{2.5}{\micro\meter} imager             \\
                                                       &                                                           \\
    \bf IRCAM3                                         & \bf UIST                                                  \\
    \smaller[2] 1--\SI{5}{\micro\meter} imager         & \smaller[2] 1--\SI{5}{\micro\meter} imager/spectrometer/IFU   \\
    \end{tabular}}
}

\headerbox{Data Reduction}{name=pipeline,column=2,span=1,below=intro,above=bottom}{
    \raggedright

    \begin{itemize}
    \item Plan to generate reduced data products using the ORAC-DR pipeline.

    \begin{itemize}
    \item ORAC-DR already been used for these instruments (except CGS3).
    \end{itemize}

    \item Will use similar infrastructure at CADC as is already
    used in the JCMT Science Archive.

    \item Reduce a complete night at once allowing
    ORAC-DR to find the required calibrations.

    \item Currently testing the ORAC-DR on a range of data.
    For observations pre-dating its introduction:
    \textsc{\huge insert date here?}

    \begin{itemize}

    \item Some header changes can be handled by updating the
    \texttt{Astro::FITS::HdrTrans} module.

    \item Other information such as grouping and recipe names
    must be provided, based on the intermediate database.

    \end{itemize}
    \end{itemize}

    \vspace{1.5in}
}

\headerbox{Example Pipeline Products}{name=example,column=0,span=2,below=intro,above=bottom}{
    \raggedright


    IRCAM3 image of HH\,111 from 1997:

    \begin{tabular}[t]{ll}
    \includegraphics[scale=0.5,angle=270]{../gfx/hh111}
    &
    \parbox[t]{1.5in}{
    \raggedright
    \begin{itemize}[topsep=0pt]
    \item
    Separate files merged.

    \item
    Grouping and recipe headers deduced.

    \item
    Processed with \texttt{JITTER\_SELF\_FLAT} ORAC-DR recipe.
    \end{itemize}
    }
    \end{tabular}
}

\headerbox{Generation of CAOM-2 Metadata}{name=ingestion,column=1,span=2,below=etransfer, above=pipeline}{

\begin{tikzpicture}[remember picture]
\draw node[text width=2.5in, text ragged] (caom2descr) {
    \vspace{-2ex}
    \begin{itemize}
    \item Metadata ingestion process split into two parts:

    \begin{itemize}
    \item Read headers from data files into an intermediate database.

    \item Write XML files using the \texttt{pyCAOM2} library, and send to the repository.
    \end{itemize}

    \item Using \texttt{MongoDB} for the intermediate database allowed the
    headers to be stored without prior knowledge of their structure.

    \end{itemize}

    Size of header database: 3.8\,GiB
};
\draw node[text width=1.5in, text ragged, font=\tiny, anchor=north west, xshift=2ex, yshift=-2ex] at (caom2descr.north east) (mongo) {
{\normalsize \bf MongoDB JSON}
\smaller \begin{tabbing}
\{ \\
\quad \quad \= \textcolor{listingfaint}{"\_id" :      ObjectId("5176688af9f5db5f13b83ee0"),} \\
            \> "group" :    null, \\
            \> "filename" : "\tikz[baseline]{\node[anchor=base,fill=yellow,text width=] (mongofile) {\smaller i19971214\_00010};}.sdf", \\
            \> "headers" : [ \\
            \> \quad \quad \=   \{ \\
            \>             \> \quad \quad \= "AMSTART" :  \quad \= \tikz[baseline]{\node[anchor=base,fill=yellow,text width=] (mongoamstart) {\smaller 1.047529};}, \\
            \>             \>             \> "AMEND" :          \> \tikz[baseline]{\node[anchor=base,fill=yellow,text width=] (mongoamend) {\smaller 1.047529};}, \\
            \>             \>             \> "INSTRUME" :       \> "IRCAM3", \\
            \>             \>             \> "MODE" :           \> "\tikz[baseline]{\node[anchor=base,fill=yellow,text width=] (mongomode) {\smaller ND\_STARE};}", \\
            \>             \>             \> "OBJECT" :         \> "\tikz[baseline]{\node[anchor=base,fill=yellow,text width=] (mongoobject) {\smaller HH111};}", \\
            \>             \>             \> "OBSTYPE" :        \> "\tikz[baseline]{\node[anchor=base,fill=yellow,text width=] (mongotype) {\smaller OBJECT};}", \\
            \>             \>             \> \textit{// \ldots} \\
            \>             \> \}, \\
            \>             \> \{ \\
            \>             \>             \> "INTNUM" :   \> 1, \\
            \>             \>             \> "RUTEND" :   \> 10.92177, \\
            \>             \>             \> "RUTSTART" : \> 10.90462, \\
            \>             \>             \> \textit{// \ldots} \\
            \>             \> \} \\
            \> ], \\
            \> "obs" :      \tikz[baseline]{\node[anchor=base,fill=yellow,text width=] (mongoobs) {\smaller 10};}, \\
            \> "rawfiles" : [ \\
            \>             \> "19971214/odir/o971214\_10.sdf", \\
            \>             \> "19971214/idir/i971214\_10\_1.sdf" \\
            \> ], \\
            \> "utdate" :   "\tikz[baseline]{\node[anchor=base,fill=yellow,text width=] (mongodate) {\smaller 19971214};}", \\
\}
\end{tabbing}
};
\draw node[text width=2.0in, text ragged, font=\tiny, anchor=north west] at (mongo.north east) {
{\normalsize \bf CAOM-2 XML Representation}
\smaller \begin{tabbing}
<\textcolor{listingfaint}{caom2:}Observation> \\
\quad \quad \= <\textcolor{listingfaint}{caom2:}collection>\textbf{UKIRT}</\textcolor{listingfaint}{caom2:}collection> \\
            \> <\textcolor{listingfaint}{caom2:}observationID>\tikz[baseline]{\node[anchor=base,fill=yellow, text width=] (xmlid) {\smaller \bf i19971214\_00010};}</\textcolor{listingfaint}{caom2:}observationID> \\
            \> <\textcolor{listingfaint}{caom2:}metaRelease>\tikz[baseline]{\node[anchor=base,fill=yellow, text width=] (xmldate) {\smaller \bf 1998-12-14T00:00:00.000};}</\textcolor{listingfaint}{caom2:}metaRelease> \\
            \> <\textcolor{listingfaint}{caom2:}sequenceNumber>\tikz[baseline]{\node[anchor=base,fill=yellow, text width=] (xmlobs) {\smaller \bf 10};}</\textcolor{listingfaint}{caom2:}sequenceNumber> \\
            \> <\textcolor{listingfaint}{caom2:}algorithm> \\
            \> \quad \quad \= <\textcolor{listingfaint}{caom2:}name>\textbf{exposure}</\textcolor{listingfaint}{caom2:}name> \\
            \> </\textcolor{listingfaint}{caom2:}algorithm> \\
            \> <\textcolor{listingfaint}{caom2:}intent>\textbf{science}</\textcolor{listingfaint}{caom2:}intent> \\
            \> <\textcolor{listingfaint}{caom2:}target> \\
            \>             \> <\textcolor{listingfaint}{caom2:}name>\tikz[baseline]{\node[anchor=base,fill=yellow,text width=] (xmlobject) {\smaller \bf HH111};}</\textcolor{listingfaint}{caom2:}name> \\
            \> </\textcolor{listingfaint}{caom2:}target> \\
            \> <\textcolor{listingfaint}{caom2:}type>\tikz[baseline]{\node[anchor=base,fill=yellow,text width=] (xmltype) {\smaller \bf object};}</\textcolor{listingfaint}{caom2:}type> \\
            \> <\textcolor{listingfaint}{caom2:}telescope> \\
            \>             \> <\textcolor{listingfaint}{caom2:}name>\textbf{UKIRT}</\textcolor{listingfaint}{caom2:}name> \textit{<!-- \ldots{} -->} \\
            \> </\textcolor{listingfaint}{caom2:}telescope> \\
            \> <\textcolor{listingfaint}{caom2:}instrument> \\
            \>             \> <\textcolor{listingfaint}{caom2:}name>\textbf{IRCAM3}</\textcolor{listingfaint}{caom2:}name> \\
            \>             \> <\textcolor{listingfaint}{caom2:}keywords> \\
            \>             \> \quad \quad \= pol=false magnifier=out detector\_mode=\tikz[baseline]{\node[anchor=base,fill=yellow,text width=] (xmlmode) {\smaller nd\_stare};} \\
            \>             \> </\textcolor{listingfaint}{caom2:}keywords> \\
            \> </\textcolor{listingfaint}{caom2:}instrument> \\
            \> <\textcolor{listingfaint}{caom2:}environment> \\
            \>             \> <\textcolor{listingfaint}{caom2:}elevation>\tikz[baseline]{\node[anchor=base,fill=yellow,text width=] (xmlelevation) {\smaller \bf 72.674337141};}</\textcolor{listingfaint}{caom2:}elevation> \\
            \> </\textcolor{listingfaint}{caom2:}environment> \\
            \> <\textcolor{listingfaint}{caom2:}planes> \textit{<!-- \ldots{} -->} </\textcolor{listingfaint}{caom2:}planes> \\
</\textcolor{listingfaint}{caom2:}Observation>
\end{tabbing}
};

\begin{scope}[on background layer]
\draw[->,>=stealth,green] (mongofile.east) to[out=0,in=180] (xmlid.west);
\draw[->,>=stealth,green] (mongoobject.east) to[out=0,in=180] (xmlobject.west);
\draw[->,>=stealth,green] (mongotype.east) to[out=0,in=180] (xmltype.west);
\draw[->,>=stealth,green] (mongomode.east) to[out=0,in=100] (xmlmode.north);
\draw[->,>=stealth,green] (mongoobs.east) to[out=0,in=270] (xmlobs.south);
\draw[->,>=stealth,green] (mongodate.east) to[out=0,in=270] (xmldate.south);

\node (mongobrace) at ($(mongoamstart.east)!0.5!(mongoamend.east) + (1ex,0)$) {\textcolor{green}{\}}};
\draw[->,>=stealth,green] (mongobrace.east) to[out=350,in=120] (xmlelevation.north);

\end{scope}
\end{tikzpicture}
}

\end{poster}
\end{document}
